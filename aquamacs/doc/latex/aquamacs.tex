\documentclass[11pt,letterpaper]{article}
\usepackage[applemac]{inputenc}
\usepackage{graphicx} 
\usepackage{html}
\usepackage{rotating}
\usepackage{palatino}
% \author{David Reitter and Kevin Walzer}
\title{Aquamacs User Help}
\begin{document}
\maketitle

\tableofcontents
\pagebreak



\section{Aquamacs Emacs: a User-friendly Emacs Distribution}

Aquamacs is an freely-available Aqua-native build of the powerful
Emacs text editor
(\url{http://www.gnu.org/software/emacs/emacs.html}). By
``Aqua-native,'' we mean more than just the fact that this version of
Emacs runs as a standard OS X application. Aquamacs features extensive
customization that enables it to conform better with Apple's standard
Human Interface Guidelines (HIG) than standard versions of the editor
do.

Emacs is a text editor of legendary power and configurability, but it
also has an enormously complex interface that, while consistent across
platforms, is usually at odds with the specific interface conventions
of the particular platform on which it is being used. The original GNU
version of Emacs for the Mac, called Carbon Emacs, is no different. 

Aquamacs Emacs implements the standard OS X keyboard
shortcuts and other interface conventions, integrating Emacs into the
Aqua environment to a far greater degree than other versions of
Emacs. This allows Mac users who might be unfamiliar with Emacs'
complex standard interface to harness its amazing editing power in a
familiar way. 

\begin{figure}
\centering
{\includegraphics[width=5in]{aquamacs-screenshot.png}}
\caption{Aquamacs combines the legendary power of Emacs with
  user-friendly customizations to provide a more Aqua-specific user experience.}
\label{aquamacs-screenshot.jpg}
\end{figure}

You can always download the latest version of Aquamacs from the
project home page, \url{http://aquamacs.org}. The direct download link is \url{http://aquamacs.org/cgi-bin/download.cgi}. Just
download the disk image (DMG), move the Aquamacs application bundle to
your hard drive, and launch.  


This documentation aims to introduce Aquamacs to novice users of
Aquamacs Emacs, to help them get started with this powerful text editor. The
documentation also aims to introduce Aquamacs to experienced users of
Emacs, who may find aspects of its interface inconsistent with their
experience.

The Aquamacs documentation will focus on the following areas:

\begin{itemize}
\item What's New in this Release
\item Tutorial: Aquamacs for Beginners
\item In-Depth: The Aquamacs Interface
\item Aquamacs for Emacs Veterans
\end{itemize}

Our hope is that using Aquamacs will be a rewarding experience both
for new users, who come to appreciate the power of Emacs without the
steep learning curve, and for experienced Emacs users, who may find
Emacs' integration into the Aqua environment an unexpectedly
pleasant surprise.

\subsection{Terminology in this Manual}

GNU Emacs uses a terminology that is different from what users of modern,
graphical environments are used to. Windows become \emph{frames}, and
documents are held in \emph{buffers}. We will concentrate on the Emacs
terminology in this manual in places where this is not confusing.

Aquamacs Emacs is an extensive \emph{distribution} of GNU Emacs with
modified defaults -- almost up to the point where it could be called a
\emph{fork} - a completely new program. However, Aquamacs always
contains the latest version of GNU Emacs. It is useful to understand
the relationship between Emacs and Aquamacs. In this manual, we will
use the term \emph{Emacs} to refer to the core which is used on many
different operating systems and built and distributed by the GNU
Project. We will use the term \emph{Aquamacs Emacs} (or just
\emph{Aquamacs}) to refer to the present implementation.



\section{Tutorial: Aquamacs for Beginners}

\subsection{What Makes Aquamacs Like Other Text Editors} 
When you first launch Aquamacs, you will see that it is like many
other text editors such as Bare Bones Edit, Dreamweaver, or similar programs: you can type text, cut and paste text, and save
and close a file using the menubar or standard OS X keyboard shortcuts
(Apple-S for save, Apple-X for cut, Apple-V for paste, and so on). If you are writing one of the many text formats that
is supported by Aquamacs, such as HTML, you will also note Aquamacs'
use of \textit{syntax coloring,} which sets certain parts of the
text---such as HTML markup---in a different color than the text
content. This makes editing the text and adjusting the markup easier.

\subsection{What Makes Aquamacs Emacs Different from Other Text
  Editors}
If you look at some of the menu items and keyboard shortcuts, you will
see some of the features that make Emacs different from other text
editors. Although Aquamacs has been designed to present many of these
features in an Aqua-friendly way, it does not hide these
features. Aquamacs is a complete editing environment.

\begin{itemize}

\item \textbf{Sophisticated text processing.} Aquamacs features text
  editing capabilities that go far beyond the average text editor. For
  instance, Aquamacs features several kinds of search and replace: it
  can replace text incrementally, it can search and replace text by
  complex patterns of characters (regular expressions) and not just by
  word matching, and so on. Aquamacs also features support for
  virtually every kind of text file imaginable: computer code such as
  C/C++, HTML, \LaTeX, XML, and other formats. 
\item \textbf{Buffers.} One of the features that makes Emacs such a productive editing
environment for experienced users is the concept of \textit{buffers.}
A buffer is, simply said, a document that is being edited. It can be
displayed in any window, and you can even display a buffer twice. But
buffers don't just hold text files. They can hold messages from a
program that's running, they can be shown in a window that you can use to
actually send commands that Aquamacs executes, and other functions. The
Buffers menu in Aquamacs Emacs allows you to switch quickly between
windows, to send execute or preview the code you are writing with a
couple of keystrokes, and to monitor logs of commands you are
executing. 
\item \textbf{Integration with additional tools.} Aquamacs' ``Tools''
  menu provides access to file comparison and version control,
  compiling and debugging of program code, the ability to read e-mail
  and newsgroups, and more.

\end{itemize}

In addition to its large number of features, Aquamacs also defines
some interface terms differently than other OS X applications. See
Table \ref{tab:terms} for more information.


\begin{table}[t]
\begin{center}
\begin{tabular}{|c|c|}
\hline  OS X Term & Emacs Term  \\ 
\hline  Window &  Frame \\ 
\hline  Tab/pane  &  Window \\ 
\hline Document &  Buffer \\ 
\hline Cursor  & Point \\
\hline Mouse pointer & Pointer \\
\hline Keyboard shortcut &  Key (binding)\\ 
\hline 
\end{tabular} 
\caption{Key Emacs terms and their Apple counterparts.}
\label{tab:terms}
\end{center}
\end{table}

This list provides just a small sample of the functionality
available in Emacs. Aquamacs' customizations make Emacs much easier to
learn; it is possible to get started and become productive
quickly. However, harnessing all of Emacs' power, even with assistance
from the familiar Aqua user interface, will take time.



\section{In Depth: The Aquamacs Interface}

In this section, we will walk through the Aquamacs interface step by
step, and will introduce relevant points about how Aquamacs Emacs solves
particular editing problems in a distinctive way. We will specifically
emphasize the functions that Aquamacs offers as opposed to GNU Emacs,
which readers might be familiar with anyways.
 
\subsection{File}
The File menu includes basic operations for opening, closing, and
printing files. Opening and saving files uses standard Mac keyboard
shortcuts (Apple-O, Apple-S), and uses standard Aqua dialog boxes. 

You can create a new buffer: Aquamacs allows you to choose from a list
of recently and commonly used editing modes.  A similar function lets
you change the current editing mode.

You can also open a directory. When called by mouse from the menubar,
this brings up a standard Aqua dialog box. When called up with a
keyboard command (Control-x d), it brings up a directory name in the
``minibuffer'' (small space for commands at the bottom of the main
window, or frame).

Printing is fully supported, although no ``Page Setup'' option is
available at this point.  You will see a print screen (in the system's
Preview.app) from where you can select a printer and the usual
options.  This may take a few seconds to appear with very large
buffers. (Printing of embedded images such in LaTeX Preview mode is currently not supported.)

You may export your buffer to PDF or to HTML format. This operation
preserves all the formatting, including any word-wrapping - so you may
want to choose a font and resize the frame so that the lines are
wrapped in the intended way (if Soft Wrapping is used).

Printing and PDF export are in US Letter format at this point.

\subsection{Edit}
The Edit menu is the heart of Aquamacs' textual wizardry. Aquamacs supports
all customary editing functions, such as cut, copy, paste, and simple
search and replace. In Aquamacs these basic functions are supported by
standard Aqua keyboard shortcuts. There is a great deal more
functionality, however, than the average text editor. For instance,
Aquamacs Emacs allows you to go to a specific line number in the file you are
editing, to the top or bottom of the buffer, and so on. It also supports searches with \textit{regular expressions,}
which are sophisticated text patterns that go beyond simply matching a
specific set of characters (or ``string''). Aquamacs Emacs also stores more
than twenty of the most recently-copied items on the clipboard, and
these are accessible from the menu in case you need to paste these
items again.  The Edit menu also supports ``bookmarks,'' a feature
that allows you to save your place in a specific file. 

Aquamacs contains spell-checking functions; however, either the extra
packages ``Aspell'' or ``Ispell'' need to be installed. The
recommended way to do that is to install ``CocoAspell'', which is a
Mac variant of the Aspell program. It comes with a dictionaries for
English, but a variety of other dictionaries is available.


\subsection{Options}
The Options menu is where you can easily customize your
settings. The options that you can configure include syntax coloring,
matching of parentheses (useful for text markup that depends on open
and close brackets), how to display buffers and frames, color theme, fonts, and other
settings. For more information on fonts, please see Section \ref{Look}.

If you want to delve deeply into customizing Aquamacs, select
``Customize Emacs:Top-Level Customization Group'' in the Options
menu. A new frame titled ``Customize Group: Emacs'' will open. Scroll
down and find the ``Aquamacs Group'' listing, and push the ``Go to
Group'' button. This will open a new frame with all of the
configuration options for Aquamacs Emacs---deep-level customization that can be implemented by Aquamacs power users in the Emacs scripting language, elisp---and each option is documented there. Table \ref{tab:variables} displays a complete list of the options. 

\subsubsection{Using the Options key and more...}

The keyboard has always been the essential user interface to Emacs -
it's what makes Emacs so efficient as an editor for daily
tasks. There are keyboard bindings for pretty much every task. Many
bindings involve pressing the ``Meta'' modifier key - it's a key just
like Control or Shift, which all go together with another (normal)
key. ``Meta'' has only really existed on Unix keyboards long time ago
-- nowadays, computers have other keys instead. Therefore, you will
need to press another key on your Macintosh keyboard. \emph{By default, this
is the Option key}, but you can use the ESC key instead (in this case,
you may press ESC first, then the rest of the keys).

\paragraph{Inputting characters with the Option key on non-English
  keyboards.} In most non-English keyboard layouts, the Option key
also serves to input characters such as $\{$ or $\backslash$ or
@. Using Option as Meta would inhibit you from inputting those
characters. You have two options to get around this. \emph{Either},
you deselect ``Option Key for Meta'' in the Options menu (under
``Option Key''), in which case you will have to use ESC for Meta,
\emph{or} you toggle back and forth between the modes using Command-;,
\emph{or} you use Option for Meta, but enable one of the emulation
modes provided in the same menu under ``Option Key''. This will allow
you to input some common characters with the Option key on some common
keyboard layouts.

Alternatively, Aquamacs allows you to use another modifier key -- such
as the Function key on laptops. The customization group ``Aquamacs''
contains appropriate settings.

\subsubsection{Languages of the World - Dealing with Different Character Sets}

Computer keyboards were designed to input text in languages with a
small character set.  Aquamacs Emacs lets you use your keyboard with a
variety of \emph{ input methods } in order to input text in languages
with many characters, among them Asian languages.
 
By default, Aquamacs uses the standard Mac user interface to input
characters, called ``System Input Method''.

The Multilingual Environment, present in GNU Emacs, provides a number
of predefined input methods.  Aquamacs Emacs extends this: On Mac OS
X, because of the particular ways of user interface and unicode
handling, these standard Emacs language environments are not enough to
read and write some international languages.  One needs to set
additional parameters, especially, the language specific coding
systems.
 
You can find these new language environments (Korean, Japanese,
Chinese-GB and Chinese-BIG5) -- among other ones -- under "Aquamacs
Multilingual Environment". 

\subsubsection{The Right Look: Colors and Fonts for Frames and Modes}
\label{Look}
\label{frame-appearance-styles}
Aquamacs allows you to alter specific features of the current frame
via functions in the Options menu. ``Show Color Themes (this
Frame)...'' will let you choose a pre-defined combination of colors
and fonts. ``Show Fonts (this Frame)...'' gives you a choice of
pre-defined fonts to use.

 \emph{ These settings only apply to the current frame.} To make them
  ``stick,'' use the function ``Frame Appearance Styles / Use current style as default.'' Then,
  future frames will open {\em by default} with the new colors and fonts
  chosen. This applies to all visual frame-settings that a mode or you
  as a user have chosen using Emacs' configuration system. The choice
  of a default style will stick until you restart Aquamacs Emacs.

  Aquamacs Emacs also offers you to pick a style specific to the
  current mode. For example, you can use different settings when you
  are editing C or LaTeX, then when you are editing a text file. That
  is what the function ``Use current style for [current] mode'' is
  for. Again, the setting will stick for newly opened frames or or
  whenever you newly use a given mode, until you restart Aquamacs
  Emacs.

  \textbf{Note that mode-specific styles will override any default
    style.} If you'd like to use the same style for all frames, you
  should use ``Use current style as default.'' and delete all the
  mode-specific frames.

If you're more of an Emacs expert and you'd like to configure things
manually by means of elisp or the customization buffer
(``default-frame-alist'', etc.), just disable the ``Frame Appearance
Styles'' option. You can then use the customization options
``default-frame-alist'' and ``initial-frame-alist'' as documented in the
Emacs manual.

Finally, you may want to save your settings so they will stick even
when you restart Aquamacs Emacs. To do so, use the function ``Save Options.''


\begin{figure}
\centering
{\includegraphics[width=5in]{theme.jpg}}
\caption{Aquamacs with a custom style applied.}
\label{theme.jpg}
\end{figure}





\subsection{Buffers}
The Buffers menu allows you to navigate through the windows/frames
that you may have open. Note that a ``buffer'' is not synonymous with
a window or frame, in that you can split a frame and have more than
one buffer contained within. (Multiple windows/frames is a feature of
Aquamacs; standard Emacs does not support this.) In addition to
standard frames that display open files, there are a few other
important buffer types. One is the ``scratch'' buffer, which is simply
a buffer to type notes into; this can also be the starting point for a
file to save, and a buffer to type configuration commands for Aquamacs
(an advanced feature). Another is the ``Messages'' buffer, which
displays a log of output from Emacs commands and
operations. Finally, there is the ``info'' buffer, in which Aquamacs Emacs
displays built-in user help, tutorials and other documentation in
Emacs' ``info'' format.

 
From the Buffers menu, you can also open a new ``frame,'' or window, or
split the open window into two separate ones. The keyboard
shortcuts for these commands are the traditional Emacs ones (see the menubar).



\begin{figure}
\centering
{\includegraphics[width=5in]{buffers.jpg}}
\caption{Using multiple buffers and windows in Aquamacs Emacs is
a powerful tool for enhancing productivity.}
\label{buffers.jpg}
\end{figure}

\subsection{Tools}
The Tools menu provides access to a variety of functions, including
integration with version control systems, running shell commands,
searching external files for text, and compiling and debugging
software code. The Tools menu also provides access to newsgroups and
e-mail.\footnote{The functionality
provided in the Tools menu is, to say the least, diverse, and is part
of the attraction of Emacs for a large number of users---particularly
advanced users. It is not necessary to use Aquamacs as an e-mail
client to appreciate its considerable power and utility, however.}

\subsection{Help}
The Help menu contains a wealth of information about
Aquamacs Emacs and GNU Emacs. Except for the help provided in this document,
Aquamacs' Help menu also provides documentation of the Mac-specific
customizations with the present User Manual. The general Emacs
user help is comprehensive and detailed to the point of possibly
overwhelming the inexperienced user. The beginner should definitely
start with the Emacs tutorial contained in the Help menu. While geared
toward the traditional Emacs interface instead of the OS X Aquamacs
version, the tutorial is a good introduction to Emacs' unique
capabilities. And, as you gain more experience, you will appreciate
the depth of the Emacs documentation. 

\section{Aquamacs for Emacs Veterans}

While experienced users of Emacs on other platforms can continue to
use all the key combinations to which they are accustomed, we
recommend that they use the Aquamacs-specific conventions to get the
most benefit from the applications.  Many of the
standard Emacs behaviors and interface conventions have been modified
in Aquamacs in the interest of providing a more Aqua-native
experience. In this section, we discuss some of the ways that Emacs
conventions are mapped to Aqua conventions, and outline some advanced ways that users can modify Aquamacs to their specific preferences.

\subsection{Keyboard Shortcuts}
Emacs has a well-defined set of keyboard shortcuts, which Aquamacs
revises to accomodate OS X conventions. See Table \ref{tab:shortcuts}
and Table \ref{tab:command} for details.



\begin{table}[t]
\begin{center}
 \begin{tabular}{|l|l|l|}
\hline  \textbf{Shortcut} &  \textbf{Elisp Command} &   \textbf{Function}\\ 

\hline Apple-N &new-frame-with & Create new buffer\\ &-new-scratch & \\

\hline Apple-O & mac-find-file- & Open a file\\ & other-frame & \\

\hline Apple-Shift-S & mac-save-file-as & Save as\\
 
\hline Apple-A & mark-whole-buffer & Select all text\\

\hline Apple-V & yank & Paste text\\

\hline Apple-C & clipboard-kill-ring-save & Copy text\\

\hline Apple-X &  clipboard-kill-region & Cut text\\

\hline Apple-S &  save-buffer & Save file\\

\hline Apple-L &  goto-line & Go to specified line\\

\hline Apple-F & isearch-forward & Search\\

\hline Apple-G &  isearch-repeat-forward & Repeat search\\

\hline  Apple-W &  intelligent-close & Close selected window\\

\hline Apple-M & iconify-or-deiconify- & Minimize window \\ & frame &  to the Dock\\

\hline Apple-. & keyboard-quit) & Keyboard quit\\

\hline Apple-, & customize & Show Customization Buffer\\

\hline Apple-; & toggle-mac-option- & Change Option key function\\ & modifier &\\

\hline Apple-{ / } / '& (un)comment-region-or-line & Comment out or in the \\ & current line or region if marked  &\\

\hline Apple-Backspace & kill-whole-visual-line & Deletes the current line \\
\hline Apple-Delete & kill-visual-line & Deletes the remainder of the current line \\

\hline Apple-Q & save-buffers-kill-emacs & Save file, exit program\\

\hline Apple-Z & undo & Undo\\

\hline Apple-Shift-Z &  redo & Redo\\

\hline 
\end{tabular} 
\caption{Aqua-specific keyboard shortcuts implemented in Aquamacs. }
\label{tab:shortcuts}
\end{center}
\end{table}


\begin{table}[t]
\begin{center}
 \begin{tabular}{|c|c|}
\hline \textbf{Emacs Command Key} & \textbf{Aquamacs / Mac Command Key}\\
\hline C-* & Control-*\\
\hline A-* & Apple-*\\
\hline M-* & Option-*\footnote{See setting in Options menu.}
(or Esc)\\
\hline 
\end{tabular} 
\caption{Aqua-specific command keys implemented in Aquamacs.}
\label{tab:command}
\end{center}
\end{table}


\subsection{Customizing Aquamacs}
One of the distinguishing features of Emacs is the degree to which it
can be customized by the end user. Emacs includes its own internal
scripting language, elisp, which allows the user to customize such
things as keyboard shortcuts, window settings, fonts, and more. The
Aquamacs customizations themselves are implemented in elisp.\footnote{The Aquamacs
customizations are stored in elisp files in the application bundle. It
is possible to modify these files directly, but we discourage this
practice and provide no support for it.}


We recommend that user customizations be placed in specific locations:

\begin{itemize}
\item /Library/Preferences/Emacs/Preferences.el: Preferences for all Carbon  Emacs installations
\item /Library/Preferences/Aquamacs Emacs/Preferences.el: Preferences for  Aquamacs and for all users
\item /Users/username/Library/Preferences/Emacs/Preferences.el: User-specific  preferences for all Carbon Emacs installations
\item /Users/username/Library/Preferences/Aquamacs Emacs/Preferences.el: User-specific  preferences for Aquamacs
\end{itemize}

If in doubt, use the last option:
\begin{itemize}
\item /Users/username/Library/Preferences/Aquamacs Emacs/Preferences.el
\end{itemize}

This replaces the usual ~/.emacs file, which is still loaded for  compatibility.

Below are some specific customization options and groups that may be of interest:

\begin{itemize} 
\item \textbf{Frame.} Aquamacs opens new files (and other buffers) in
  new frames. That is usually more convenient and allows you to use
  the graphical user interface of today's computers, which did not
  exist when Emacs was conceived almost three decades ago. If you do
  not like this behavior, perhaps because you are used to traditional
  Emacs, just deselect ``Display Buffers in Separate Frames'' in the
  Options menu and save your choice with ``Save Options.'' (The
  associated global minor mode is called
  ``one-buffer-one-frame-mode''.)

\item \textbf{Frame Appearance Styles.} These styles provide a
  convenient way to tie bundles of frame parameter and faces to
  specific major modes so that every frame showing a buffer in a
  particular major mode is styled that way. Such frame parameters can
  be the background color, they can be the faces that define syntax
  coloring. They also comprise settings such as if and where there is
  a fringe, or a tool-bar (icons on top). Such settings always
  override any global choices made with, for instance,
  ``default-frame-alist'' or ``tool-bar-mode''. Because the style
  always applies to the frame and not the window (an Emacs
  limitation), this mode makes most sense together with
  ``one-buffer-one-frame-mode'', where only one (main) buffer is
  usually shown in a frame.

\item \textbf{mac-option-modifier.} This is  the modifier to use for the Mac alt/option key.  The value can
be alt, hyper, or super for the respective modifier.  If the value is
nil then the key will act as the normal Mac option modifier, and the option
key can be used to compose characters depending on the chosen Mac keyboard
setting. 

\item \textbf{Additional customization.} Of course, Aquamacs Emacs
  offers you almost all the customization  possibilities that Emacs
  has. Under ``Customize Emacs,'' you will find  a sub-menu that
  allows you to browse the vast space of customization
  settings. Beware: some of them are complex and not easy to
  understand. If you would like to tinker with some general Aquamacs-specific
  behavior, you can customize the group ``Aquamacs.'' 

\item \textbf{Want some GNU Emacs 22 behavior back?} In most cases
  that represents no problem. Aquamacs provides a customization group called ``Aquamacs-is-more-than-Emacs'' which contains all the settings whose default values differ from the ones in GNU Emacs (22) or the appropriate package. We recommend that you tinker with these one at a time and check their (often wider-than-you-think-reaching, sometimes detrimental) effects.

Also, see Table \ref{tab:variables} for a partial list of such settings.

\begin{table}[t]
\begin{center}
 \begin{tabular}{|l|}

\hline  \textbf{Variable}\\

\hline Aquamacs Set Creator Codes After Writing Files\\
\hline Aquamacs Quick Yes Or No Prompt\\
\hline Aquamacs Ring Bell On Error\\
\hline Aquamacs Known Major Mode\\
\hline Aquamacs Default Styles\\
\hline Aquamacs Buffer Specific Frame Styles\\ 
\hline Aquamacs Auto Frame Parameters Flag\\
\hline One Buffer One Frame\\
\hline Smart Frame Positioning Enforce\\
\hline Smart Frame Positioning Mode\\
\hline Mac Option Modifier\\ 
\hline Mac Control Modifier\\ 
\hline Mac Command Modifier\\ 
\hline Mac Function Modifier\\ 
\hline Aquamacs Known Buffer Modes \\
\hline OS X Key Mode \\
\hline OS X Key Mode Mouse-3 Behavior \\
\hline Visual Scroll Margin \\
\hline Emulate Mac ... Keyboard Mode \\
\hline
\end{tabular} 
\caption{Selection of Aquamacs-specific variables that can be customized in the Aquamacs customization group.}
\label{tab:variables}
\end{center}
\end{table}


\end{itemize}

The range of possible customizations---including restoring some of
Emacs' traditional interface conventions---is beyond the scope of this
help document. However, we provide a wiki for users to share their
modifications. See
\url{http://www.emacswiki.org/cgi-bin/wiki/AquamacsEmacs/} for more
details.

\subsection{\LaTeX\ Support}

One special feature of Aquamacs is its extensive support for the
editing of \LaTeX documents, especially Emacs' Auc\TeX\ mode. 

\begin{figure}
\centering
{\includegraphics[width=5in]{aquamacs-tex.jpg}}
\caption{Aquamacs offers extensive support for \LaTeX\ documents.}
\label{aquamacs-tex.jpg}
\end{figure}

AUCTeX comes with its own manual, accessible via the LaTeX menu (when in latex mode after loading a .tex file), ``Read the AUCTeX manual''. If you have a question, please consult the manual. If that doesn't work, or you suspect you've found a bug in AUCTeX, please turn to the AUCTeX mailing lists, accessible here: \url{http://www.gnu.org/software/auctex/mailing-lists.html}.

To access the enhanced \LaTeX\ functionality that Aquamacs offers, you
will need to install a LaTeX system. We recommend to install the
\emph{MacTeX distribution}, available at
\url{http://www.tug.org/mactex/}. Alternatively, you can install a
smaller distribution provided by Gerben Wierda
(\url{http://www.rna.nl/tex.html}. Both install packages are complete
and user-friendly. (While you can obtain \TeX\ from
other sources, the binary distributions are particularly well
supported in Aquamacs.)

\subsubsection{Enhanced Carbon Emacs Plugin}
Enrico Franconi, developer of the former Enhanced Carbon Emacs
distribution that had special support for La\TeX\ editing via its
customizations for Auc\TeX\ mode, has revived ECE as an external
plugin that can be installed on any Carbon Emacs,including
Aquamacs. At this time, the ECE plugin 1.0.1 is available at

\url{http://web.inf.unibz.it/~franconi/enhanced-carbon-emacs/} 


\subsubsection{PDFView: A   \LaTeX\ Previewer}
When editing \LaTeX files, you will often need to view the compiled
results. PDFView is a program that will automatically show the typeset
document. It is available here:

\url{http://pdfview.sourceforge.net}

An older such tool is \TeX niscope. \TeX niscope can be downloaded from 

\noindent
\url{http://www.ing.unipi.it/~d9615/homepage/texniscope.html}. 

\TeX niscope needs a bit of configuration to work well with Aquamacs. See the Wiki page at \url{http://www.emacswiki.org/cgi-bin/wiki/AquamacsTexniscope} for up-to-date configuration info. 

\section{Extending Aquamacs}

Aquamacs supports Plugins that can be packaged to provide point-and-click installers (via Apple's PackageMaker). The interface is simple.
Adding standard Emacs packages is easy as well.

\subsection{Plug-in Interface}

All files named \emph{site-start.el} anywhere in the load-path
    are loaded at startup, before the user's init files are
    loaded. For example, such a file may be placed in
    \~/Library/Application Support/Aquamacs Emacs/myPlugin. 

The load-path is automatically extended to cover all subdirectories in standard paths (see Section \ref{standardpaths}).

If a plug-in needs to execute code earlier in the initialization phase, that is, before any Aquamacs-specific code is executed (but after Emacs initilization code has been executed), this code can be provided in a file  \emph{site-prestart.el} anywhere in the load-path.

Note that plug-ins should not make any assumptions about whether a frame exists or is visible.


\subsection{Load-path}

Aquamacs is able to find configuration files and Emacs packages
from any of the following locations (load-path). 
\label{standardpaths}

{\tt
/ Library/ Application Support /Aquamacs Emacs / \\
/ Users / User Name / Library/ Preferences / Aquamacs Emacs/\\
 / Library/Application Support / Emacs / \\
 / Users / User Name / Library/ Preferences / Emacs/\\
}

The following paths are available in the load-path for individual preference settings. \emph{They should not be used if you distribute a point-and-click installer package.}

{\tt
/ Library/ Preferences /Aquamacs Emacs / \\
/ Users / User Name/Library/Preferences / Aquamacs Emacs/\\
 / Library/Preferences / Emacs / \\
 / Users / User Name / Library/ Preferences/ Emacs/\\
}

All subsequent directories are added recursively, so Aquamacs will find any path that starts below. Type {\tt C-h v load-path} for more information.

\subsection{Disabling a plug-in or a load-path}

If an (emtpy) file named \emph{.ignore} exists in a specific path, Aquamacs does not load any \emph{site-start} and \emph{site-prestart} source files in this or descending paths and does not add that or subsequent paths to the load-path. 


\subsection{Changing Aquamacs source files}

If you'd like to tinker with the Emacs lisp source files that come with Aquamacs,
you can do so quite easily. When activaten a new function or evaluate an
expression, C-x C-e is your friend. 

However, you may notice that some
of your changes aren't active after you've saved your work (into the
Aquamacs application bundle) and restart. That is because some of the
code is precompiled into the Aquamacs binary and not loaded from the
source (or .elc / .el.gz) files. To alleviate that problem, create a
file ``\~/Library/Application Support/Aquamacs
Emacs/site-prestart.el'' with the following content:

\texttt{(setq aquamacs-reload-preloaded-files t)}

This will cause to re-read the otherwise preloaded files from disk.

\section{Getting Help}

There are many options for getting help with Aquamacs.

From within Aquamacs, you can access user documentation from the menu
or from specific key combinations: C-h k (key/menu entry) brings up
help for some input items; C-h f (function) gives help for an elisp
function; autocompletion support is available with (tab); and  
C-h a brings up apropos, a search function.

For help from other Aquamacs/Emacs users, the best place
to begin is the OS X Emacs mailing
list. The searchable list archives are located at
\url{http://www.esm.psu.edu/mac-tex/MacOSX-Emacs-Digests/}. For more
information on subscribing to the list, see
\url{http://aquamacs.org}.

Another option for general Emacs help is the gnu.emacs.help newsgroup.

Apart from
answering questions at the OS X Emacs mailing list, you can also file
bug reports on Aquamacs. Use the ``Send Bug Report'' function in the
Help menu  (general Emacs bugs, too), and also our Aquamacs-specific
bug report  system at \url{http://sourceforge.net/projects/aquamacs}. 

In addition to requesting help, you can also offer it in these mailing lists. \emph{Please note that in case we have helped you with a configuration problem on the mailing list, we may ask you to write a little note in the Aquamacs Wiki so others can profit, too.} Remember: Aquamacs is an Open Source community project!

\section{Donations}

Aquamacs Emacs is a project that depends on your support. You can donate very small amounts by looking at ads, or small or larger amounts directly. Both is appreciated. Your contributions will be used towards maintaining the Aquamacs web site and further development.

\url{http://aquamacs.org/donate}

\section{Acknowledgments}

Who's behind Aquamacs?

It's a free software project, and naturally, Aquamacs could not exist
without the help of developers, designers, writers and users. The
following people have contributed code, bug reports and ideas to the
Aquamacs project at one point of the other.

Lawrence Akka; Mahn-Soo Choi; Bill Clementson;
Adrian Chromenko (icon artwork);
Nathaniel Cunningham; Jos� Figueroa-O'Farrill; 
Enrico Franconi; Terry Jones; Jean-Christophe Helary;
David Reitter (co-founder, lead developer);
Robert H. Sloan;
Kevin Walzer (co-founder, documentation); and others.

We would also like to acknowledge the contributions of these authors,
whose source code and hints on public forums have already been
integrated into the build:

Drew Adams;
Emil Astrom; 
Stefan Bruda; 
Steve Dodd;  
Massimiliano Gubinelli (Emacs icon); 
Taiichi Hashimoto;
T. Hiromatsu; 
 Kyle E. Jones; 
Paul Kinnucan (JDE); 
Eric M. Ludlam (Cedet);
 Xavier Maillard;
 Pekka Marjola;
Mitsuharu Yamamoto;
Gerd Neugebauer;
 Ovidiu Predesc; 
 Alex Schr�der;
 Mikael Sj�din;
Steven Tamm; 
 Bob Weiner;  
Milan Zamazal;
Seiji Zenitani; and many others.

Aquamacs is based on GNU Emacs, which is the work of Richard Stallman
and many other developers, in particular Andrew Choi (main GNU
Emacs-to-MacOS/OSX port). We also would like to acknowledge the work
of the many authors of the major modes included.

Last but not least, Aquamacs couldn't exist without the dedication of
its users, who donate money, report bugs and help us fix them.

\section {Contribute to Aquamacs -- Join the Fun!}

If you'd like to join us in working on Aquamacs, download the source
code from CVS and take a look. Do ask if you have questions! When you
have a patch or a less specific suggestion for improvements, send it
to the development mailing list.

We're a very open team. Everyone is allowed to suggest modifications
to every piece of code. There is no ``ownership'' of code, so please
join us to make Aquamacs even better!

\section {Nightly Development Builds}

Ready to install, the Aquamacs project offers nightly built binary
packages that contain the latest in Aquamacs development. These builds
are not considered ``stable'', but experimental.

See \url{http://aquamacs.org/nightlies.shtml} for downloads.


\section {Obtaining the Source Code}

You may download the source code for each release of Aquamacs, or
download the source of the current development version (CVS) from
\url{http://aquamacs.org}.

The author offers  to provide you with the source code for a time period of
at least three years from the date of release. Source code may be
obtained from Sourceforge.Net (\url{http://sourceforge.net/projects/aquamacs
}) or, if this is impossible, from the author directly.

\section {Licenses}

Aquamacs: the Emacs distribution, (c) 2005,2006,2007 by Free Software
Foundation, Inc., and David Reitter. Aquamacs is licensed under the
terms of the GNU General Public License (Version 2). 
For details of this license,
please see \url{http://www.fsf.org/licensing/licenses/gpl.html}. The Aquamacs
documentation is licensed under the terms of the GNU Free
Documentation License. For details of this license, please see
\url{http://www.gnu.org/copyleft/fdl.html#SEC1}.

The license does not apply to the toolbar icons contained in the
Aquamacs .app bundle. They may be used and redistributed within
Aquamacs, but not separately or in another application.


\section{What's New in This Release}
\subsection{Changes---1.0}


\begin{itemize}


\item Java support has just gotten a lot better. JDEE is a complete,
	integrated development environment for Java in Emacs. It supports
	building, source-level debugging, code generation and much
	more. Aquamacs now contains the latest version of JDEE, configured
	to work with the Java SDK installed via Apple's Developer
	Tools. Visit http://jdee.sunsite.dk for more information.

\item Toolbar-Icons are new. This artwork is by Adrian Chromenko
        and David Reitter. Toolbars have also been revised.

\item Spell-Checking can now be found in the Edit menu, not in
	Tools. The "Check As You Type" function (flyspell-mode) will
	automatically check the buffer contents (unless it's a big buffer)
	when first enabled.
	
\item Spell-Checking now supports the popular cocoAspell package,
	which includes English language dictionaries and the GNU Aspell
	spell-checking library. Additional dictionaries may be installed.
	
\item Formatting (filling) text with M-q (Option-q) is a bit more
	convenient now: when a region is selected, the command applies to
	the whole region and not just to a paragraph.
	We also provide M-Q (Option-Shift-q) to "unfill" a paragraph or a
	region, which comes in handy when wrapped text needs to be
	converted to soft-wrapped text, or to copy&paste it into other applications.
	
\item one-buffer-one-frame-mode is more compatible to the normal Emacs
	paradigm of one-frame-many-buffers for internal technical
	reasons, but also with respect to whether newly opened frames
	(windows) are actually selected and receive the following user
	inputs. In many situations, classical Emacs shows a buffer in an
	extra window, but doesn't select it - so the user can keep
	working on the original document. Aquamacs will now do the same -
	it will raise the appropriate frame, but not select it. If you
	would like to make it select at all newly opened frames, use this
	in your Preferences.el:

	(setq display-buffer-reuse-frames 'select)

	Also (the technical change): Function `switch-to-buffer' will set
	the current buffer immediately without waiting for the next
	top-level event loop.)  Thanks to Lowell Vaughn for pointing out a
	seemingly unimportant bug in one of Emacs' on-board games, which
	led to the discovery of this issue.

      \item Undo as in the original Emacs is available again via
        `undo'. Apple-Z and Apple-Shift-Z are bound to Aquamacs
        undo/redo functions (now called `aquamacs-undo/redo' as they
        used to be, but seasoned Emacs users may now use
        Control-underscore to access the classical undo. The
        difference is that the old-school undo will undo it's own undo
        steps.  Suggested by Rudi Schlatte
	
\item Undo (aquamacs-undo, Apple-Z) will now restore point and mark as
	appropriate.

\item Left/Right keys will now, when text is marked (a region is
	defined), behave more Mac-like: they move to the beginning or the
	end of that region, respectively. (Rebind left/right keys in
	osx-mode-map or turn off CUA mode to get rid of this new
	behavior).

\item Apple-2 can now be used to split the window into two (like C-x
	2), Apple-1 will undo that split (like C-x 1).

\item Emulate-Mac-{German|Italian|...}-Keyboard-Modes now allow you to
	call the commands bound to the original Meta key combinations
	using the "Esc" key. For example, in the German keyboard
	emulation, pressing Option-l will give you @, while Esc l will do
	`downcase-word'.
	Also, this mode now supports searching (Apple-F) with `isearch'.
	Reported by Thomas K�ufl
	
\item Default colors in help and *Messages* frames/buffers revised.
	Suggested by Charles Bailey

\item Recently opened files that were opened from the "Recent Files"
	menu, or from an external shell with the 'open' command or
	something similar are now available in the minibuffer history that
	one gets with C-x C-f <up>.
	
\item Clicking into raised frames (windows) that are not the selected
	frame (i.e. have grey instead of red-yellow-green buttons) will
	now activate that frame.
	Patch by Yamamoto Mitsuharu

\item Creator codes for loaded files are maintained and not always
	reset to the Aquamacs / Emacs code. This way, users may use the
	Finder to open specific files with another application by default,
	but still edit the file in Aquamacs.

\item Updated AUCTeX (11.84). [M-x latex-mode].
	XeTeX users: please customize `TeX-command-list' to call XeTeX
	directly.
	
\item Haskell mode included (2.3). [M-x haskell-mode]
	
\item RubyOnRails support via emacs-rails included (0.5.3)
					[M-x rails-minor-mode, with M-x ruby-mode]

\item ACT-R support. [M-x actr-mode]
	
\item Menu Bar fixes. (Disabled menu bar entries when frames invisible.)

\item Aquamacs menu name shortened.
	Suggested by Enrico Franconi
	
\item Accessibility in OS X fixes (simulated clicks on Window elements)
	Suggested by Alexis Gallagher
	Patch by Yamamoto Mitsuharu

\item HTML files (and others) can be opened in the system's default
	browser again. This applies to C-c C-z v in html-helper-mode and
	other modes using `browse-url'.

\item  Tramp files (i.e. remote files loaded with
	/uid@host:/path/to/file syntax) are noted in the "recent files"
	history list and not discarded any longer when Aquamacs is
	restarted.

\end{itemize}

	
	Thanks for supplying build machines: Toby Blake (U Edinburgh),
	James Pringle (U New Hampshire) and Carlo Gandolfi (Milan).


\subsection{Changes---0.9.9d}


\begin{itemize}


\item  Frames (windows) now have a little icon in the title bar for the
	selected buffer, and you can drag\&drop this to other applications.
	Patch by  Mitsuharu Yamamoto

\item  Clicking the Dock icon of Aquamacs will now bring up a minimized
	frame, if there is no other frame to be shown - just like in any
	OS X application.
	Suggested by Frank Atanassow

\item  Fonts can now be selected comfortably with the OS X standard
	font selector, accessible via "Options -> Show Fonts".
	Remember  - to make a setting stick, use the "Frame Appearance
	Styles". Set the new style either for the given major mode, or as
	a default.
	Patch by  Mitsuharu Yamamoto
	
\item  C-n and C-p are now mapped to `next-line' / `previous-line'
	again, so they may be used to jump full lines (as defined by
	carriage returns) - even if the lines are wrapped around. The
	arrow keys move the cursor visually as before. 
	Suggested by Tom van Vleck

\item  Apple-` now cycles through all the frames, not just the topmost two.
	Reported by William Henney

\item  Apple-: now spell-checks the buffer, provided ispell is installed.
	Suggested by Jean-Christophe Helary
	
\item  Apple-Shift-S will now bring up a file selector rather than
	prompting for a file name in the minibuffer. If you prefer the
	Emacs-style prompt, you should use the Emacs-style key binding:
	C-x C-w.
	Suggested by Jean-Christophe Helary

\item  Emulate-Mac-Keyboard-Mode: improved support for French keyboard
	layouts
	
\item  ESS 5.3.0 included

\item  SLIME - the Superior Lisp Interaction Mode is now available with
	Aquamacs. It's pre-configured for you. Use `lisp-mode' for your
	Lisp editing, and run M-x slime to start a lisp inferior
	process. You may need to set the variable `inferior-lisp-program'
	via customization or in your Preferences.el as follows:

\begin{verbatim}
	(setq inferior-lisp-program "path to lisp")
\end{verbatim}


\item  C++ Mode is used automatically for .cp files.
	
\item  Keyboard bug-fix: certain keys could not be successfully input
	on certain keyboard layouts, most notably the key combination 
	C-c ` as used in LaTeX mode (AUCTeX) was not available on German
	keyboards.
	Patch by Taiichi Hashimoto

\item  Ispell/LaTeX bug-fix: ispell will now ignore TeX commands in
	such buffers.
	Patch by Bill Rising
	

\end{itemize}	

\subsection{Changes---0.9.9c}

\begin{itemize}

\item Bugfix: Emacs-internal Input methods work again. 

\item System-wide input methods (via the ``International'' preference
	panel) can now be used instead of the Emacs-specific ones.  Use
	Options - Aquamacs Multilingual Environment - Use System Input Method
	to turn it on or off - it's switched on by default.
	Patch by Taiichi Hashimoto
\end{itemize}	
 
\subsection{Changes---0.9.9b}

\begin{itemize}
 
\item `mac-pass-option-to-system' didn't work as intended. There may
	have been other input problems.
	
\item Switching to another buffer did not change the input focus
 	(i.e. where things you type go), even though the frame displaying
	the buffer was raised.
	Reported by Rick Taube

\item tool-bar-mode calls in Preferences.el / .emacs could bring up
	errors under certain circumstances
	Reported by William Henney

\item `bury-buffer' didn't work correctly, causing situations where a
	frame was shown and was removed immediately, for example when
	SLIME started up or, in LaTeX (AUCTeX) mode, C-c C-l was used to
	display compiler output.

\item Crash in `show mouse face' fixed
	Reported by Phil
	
\item  Russian as first application language (System Preference) does
	not cause startup problems any more.
	Reported by Vladimir Lech

\end{itemize}



\subsection{Changes---0.9.9a}

\begin{itemize}

\item  SUMMARY: More bugs and other annoyances 
	have been removed. Opening new frames (for new buffers) is more
	useful now, and frame themes aren't applied when a frame is split
	up. Options are automatically saved across sessions. A new
	customization groups for Emacs-die-hards makes it easier to change
	Aquamacs-specific customization defaults. 

\item  Intel: Aquamacs is now available as native version for Mac
	computers with Intel CPU.  Thanks to Jonathan Shea, Mike Bennett
	and Richard Koch for providing compile machines.
	
\item  Dired in one-buffer-one-frame-mode will show new
	directories and open files in the same frame.

\item  Customize will show new groups etc. in the same frame and window.

\item  One-Buffer-One-Frame-Mode ("Display Buffers in Separate Frames")
	will now show buffers such as *mail*, *grep*, *shell* in their own
	frames. This also applies if other modes open *...* buffers, such
	as ESS opening an inferior R shell, which previously didn't even
	show in a separate window. One-Buffer-One-Frame-Mode is also better
	documented now and has customization options so you can decide
	which buffers are shown in new frames.	
 	 
\item  Frame-Appearance-Themes reconfigure a frame only if just one
	window is present (or all windows show buffers in the same major
	mode). This should more effectively prevent appearance switching
	when a second window is shown temporarily.

\item  Frame-Appearance-Themes now allow the user to delete all
	mode-specific themes when a default is defined. Hints to select
	"Save Options" are more prominent.

\item  Options are saved automatically upon quitting Aquamacs (it'll
	ask you.) - customize `aquamacs-save-options-on-quit' to
	configure.

\item  Quitting Aquamacs: you won't be asked any more whether to quit
	if modified buffers exist, because at that point you've always already
	been asked to save each modified buffer.
	
\item  Aquamacs-is-more-than-Emacs is a new customization group
	containing all customization variables whose defaults are changed
	in Aquamacs compared to GNU Emacs or the relevant package.
	This is useful for Emacs veterans who want certain behaviors to be
	more similar to GNU Emacs.

\item  Bug Reporting is handled fully in the system's default mail
	client again (unless you've set up a different
	`send-mail-function')
	Suggested by Robert Sloan

\item  Mail composed with the Message package will now be routed
	through the mail client by default (`message-send-mail-function')
	and there is no "From" line. If you run postfix locally, you may
	want to do something like the following in your .emacs or Preferences.el:

\begin{verbatim}
	(setq send-mail-function 'sendmail-send-it)
	(setq message-send-mail-function 'sendmail-send-it)
	(setq mail-setup-with-from t)
\end{verbatim}

	Alternatively do the same in customization buffers.	

\item  Toolbar looks prettier now.
	Patch by Kim F. Storm
	
\item  `Frame-Appearance-Styles' is the new name for the former `Frame
	Appearance Themes, so they won't be confused with GNU Emacs new
	customization themes. A global minor mode called
	`aquamacs-styles-mode' is now used to switch them on/off.

	Some configuration variables have been renamed: 
	`aquamacs-default-styles' 
	       (was: `aquamacs-mode-specific-default-themes')
	`aquamacs-styles-mode' 
	       (was: `aquamacs-auto-frame-parameters-flag')
	`aquamacs-buffer-default-styles' 
	       (was: `aquamacs-buffer-specific-frame-themes')

\item  A-Backspace deletes the whole logical line again, not just
	the visual line. A-Shift-Backspace deletes the visual line now.

\item  Tilde ~ can now be input on German keyboards with the German
	emulation mode via Option-N.

\item  Frame-positions for individual files are now saved to/loaded
	from a file called frame-positions.el (in ~/Library/Preferences...)
	

\item  Help buffers (C-h f, C-h v) for certain Aquamacs functions could
	have links to source code at nonexistant paths.

\item  site-start.el files in the load path (plug-ins!) aren't loaded
	any longer if there is a file named .ignore in the same path.

\item  Bugfix in version 0.9.9: customizations weren't read if
	aquamacs-styles (the former "Frame Appearance Themes" were
	switched off.)
	Reported by Brendan Dixon
	

\item  Bugfix: Typing C-x C-f, then A-w made the newly opened frame go
	away, but not the "Find file" prompt, which was quite persistent,
	especially if you didn't know about C-g and the minibuffer. That
	is fixed now. 
	Reported by a number of people.
	
\item  Bugfix: Setting tool-bar-mode or changing the fringes via the
	Options menu will now affect the default frame style as well, so
	the setting will affect new frames (even if frame appearance
	styles are active). Note that if fringes or tool-bar are set in
	.emacs, the customized and saved styles will override your
	settings. 
	
\item  Bugfix: Aquamacs will not ask the user any longer whether to
	save dired buffers and other non-file-buffers upon deleting the
	frame (killing the buffer).
	Reported by Carsten Bormann

\item  Bugfix: desktop-mode should be able to restore the desktop on
	startup appropriately. 
	Reported by Rudi Schlatte
	
\item  Bugfix: page-wise scrolling (PgUp/PgDown) with long lines
	scrolls exactly one screen now

\item  Bugfix: page-wise scrolling (PgUp/PgDown) will remove the region
	as expected when the cursor moves. Shift-PgUp/PgDown will extend
	the region (and set the mark if neccessary).
	
\item  Bugfix: completing-read-multiple didn't work (icomplete+ library
	updated)   

\item  Bugfix: frames sometimes didn't open in
	one-buffer-one-frame-mode when the buffer shown was empty.
	
\item  Bugfix: In one-buffer-one-frame-mode, Apple-W deletes a window
	or kills a buffer and returns to the next buffer in the buffer
	list (i.e. the previously shown buffer) when the frame wasn't
	created for the particular window / buffer being killed. Think of
	it as a stack of buffers that are shown. Previously, Aquamacs
	always switched to *scratch*.

\item  Bugfix: Preview & Print now works with buffers in Tramp Mode, too.
	Reported by Tim Hodgson

\item  Bugfix: Fonts (not fontsets) set in default-frame-alist and
	similar variables are not filtered out any more after startup if
	they haven't been explicitly loaded (or applied to a frame, and
	thus, loaded).
	Reported by Konstantinos Efstathiou
	
\item  Bugfix: display copyright message at startup. 

\item  Bugfix: Javascript-mode exchanged for the one borrowed from
	XEmacs. This one will hopefully work better now.
	Suggested by Tom Van Vleck

\item  Bugfix: ediff-directories interferes less with one-buffer-one-frame-mode.
	
\item  Compiling: Those who build Aquamacs themselves, note that the
	make-aquamacs script will now produce a complete app bundle
	located in (emacs-path)/mac/Aquamacs. The install-aquamacs script
	only moves the bundle to /Applications. If you make any changes to
	the Aquamacs source code (except for pre-loaded code in
	site-load.el), it is usually sufficient to run 
	"copy-to-app -n". Remember to set the AQUAMACS\_ROOT and EMACS\_ROOT
	environment variables before using any of these scripts.

      \item A BIG THANKYOU goes to all users who have donated
        (\url{http://aquamacs.org/donate}) in the past months. I'd
        also like to thank all those who tested the beta versions,
        reported bugs and contributed ideas and server space:
        Nathaniel Cunningham, Peter Dyballa, Terry Jones, Jos� Figuera
        O'Farrill, Konrad Podczeck, Robert Sloan and many others.

\end{itemize}

\subsection{Changes---0.9.8}

\begin{itemize}

\item Meta/Option key: New minor modes allow users of various
	keyboard layouts (German, French, Italian, GB/US) to use the
	Option key as Meta modifier under Emacs (as it is the
	default), but still be able to input characters such as [ ] {
	} | and @. This works for common characters (not all). Some
	rarely used Emacs key bindings such as M-4 on Italian
	keyboards are overridden by these modes. Activate them via the
	Options menu ("Option Key"), or with M-x
	emulate-mac-french-keyboard-mode (substitute 'french' with
	a keyboard layout in lower case).
	The following keyboard layouts are known:

	German, Italian, Italian-Pro, French, US, British

	You may define your own mappings. (Please publish them.)
	
\item Bug reporting: Use standard GNU Emacs bug reporting facility
 	(but go through the mail client when necessary.)

\item One-Buffer-One-Frame: Change in the decision-making process
	about when to show a buffer in a new frame. Generally, all buffers
	with names in *'s (such as *vc*) will be shown in the same
	frame (as directed by `obof-same-frame-regexps'), with the
	exception of Messages, Help and Customize buffers (as listed in
	`obof-other-frame-regexps'). This means that more buffers will
	open as a "small" window inside the current frame. Please report
	back to us in case you have any unwanted results with this.

\item Special-Display-Frames such as the one with the *Help* buffer
	can now show *Completions* buffers and other ones in a second
	window inside the frame, just like other frames can.
 	(`unsplittable' property in `special-display-frame-alist').
	
\item Drag\&Drop of files in buffers with HTML/LaTeX/C modes results in
	the appropriate code (such as $<$IMG SRC=...$>$ or
	$\backslash$includegraphics$\{$...$\}$) being inserted.
	Patches by Seiji Zenitani
	
\item AUCTeX/PDF: TeX-PDF-mode is now the default via the default of
	its mode variable `TeX-PDF-mode' rather than an entry in
	`LaTeX-mode-hook', making configuring the default in the
	customization buffer (or in .emacs) easier. Updated to AUCTeX 11.82.

\item DONATE: If you like the new version, please donate so we can
	keep the development efforts going throughout 2006. Have a great
	new year!
\end{itemize}	

Bugfixes:

\begin{itemize}
\item Browsing a URL (as in Help menu functions, mouse clicks in Gnus
	etc. could fail in rare circumstances.
	Reported by Count L�szl� de Alm�sy

\item Tcsh environments will be correctly imported into Aquamacs now.
	Reported by Michael Kohlhase
	
\item AUCTeX and RSS documentation (info) are now included.
	Reported by Arthur Ogus and Jos� Figueroa-O'Farrill
	
\item Avoid the Dock when opening or automatically resizing frames in
	most situations. 

\item Up/down movements of the cursor: subtle tweak (jumping to
	temporary goal column only in series of up/down movements.)

\item `mouse-save-then-kill' for `osx-key-mode-mouse-3-behavior' set
	to now works fine for double clicks (deleting the selected region)
	Reported by Konrad Podczeck
	
\item Tcl-mode back in the "New Buffer" menu.
	Patch by Kevin Walzer

\item CPerl mode is still the default for perl files, but perl-mode is
	not an alias any more for cperl-mode.
	
\item Environment import (from the default login shell) works better
	now for users of tcsh, zsh and ksh.
	


\end{itemize}

\subsection{Changes---0.9.7}

\begin{itemize}


\item {\emph Summary:} New functions: Mac-style printing, Export to PDF and
	HTML, Spelling suggestions, New buffer in mode / change major mode
	menus, a context menu (right mouse click), speed improvements,
	nXML and AUCTeX included text display improved (international
	characters and proportional fonts), "wrong command-p" bug fixed
	and much more.

	
\item Printing: Aquamacs now supports full What-you-see-is-what-
	you-get printing. Choose 'Preview and Print' from the File menu,
	and the current buffer (or the current region, if one is marked)
	is typeset using the color theme and the syntax coloring that is
	active. You can then preview or print it with the standard Mac OS
	X printing dialogs, selecting a standard Mac printer. (Output to
	US Letter format only at this point. Sorry!)

\item Export to PDF and HTML: The current buffer (or a region if
	selected) can now be exported to a PDF or HTML file. Syntax
	coloring (and the chosen color theme except for the background
	color) are respected.
	
\item LaTeX: Aquamacs now comes with the latest version of
	AUCTeX, which will allow you to most comfortably edit LaTeX source
	code and even preview some of the results (e.g. formulas) directly
	in your document.  No need to install an additional AUCTeX
	package. 	AUCTeX already works out-of-the box.

	It is recommended that you remove any previous installations of
	AUCTeX. 
	
\item nXML: Edit XML code more comfortably with the included
	nxml-mode. It validates the document as you type. If you don't
	like the mode, the following in your Preferences.el file will
	bring the old XML mode back:

\begin{verbatim} 	
(assq-set-equal "<\\?xml " 'xml-mode 
                           'magic-mode-alist) 
\end{verbatim}

	Suggestion from Tim Bray.

\item Speed: Aquamacs is a little faster now when opening new frames,
	when frame appearance styles have been defined. 
	
\item Right-Mouse-clicks in text now bring up a context menu which
	allows you to copy/paste, search the word under the mouse in
	Google or look it up in the system dictionary and switch the buffer.

\item Control-Mouse-click is always mapped to mouse-3 (right mouse
	button) now so Apple mouse and laptop users will consistently get the same
	functionality. To do mouse-2, just hold down the Command key while
	clicking. (This does mouse-yank-at-click, a useful functionality that you
	might be used to from Linux / X11 machines!)

\item Alt-Mouse-click\&drag allows you to select a secondary
	selection. (This used to be on Command-mouse-1-drag.)
	
\item Spelling correction (flyspell) will make suggestions upon
	right-clicking on a misspelled word.

\item Fonts/Display: Aquamacs Emacs can now display an even wider
	range of characters with any font available on OS X. Some
	characters will look nicer.  
	Patches by Yamamoto Mitsuharu

\item Text Display: Aquamacs now displays text in variable-width fonts
	with much better letter-spacing. It'll look like in other OS X
	apps, and reading and editing long texts is much more fun now.
	Patches by Yamamoto Mitsuharu

\item Navigation in wrapped lines (not word wrap) is easier now: arrow
	keys will jump to the next visual line, not to the next
	buffer/file line. Also, navigation is visual now for variable
	width fonts like Lucida: jumping up or down moves to roughly the
	same place in the adjacent line going after what's displayed on
	the screen rather than counting characters (jumping to the same
	row number in the next line).  If you don't like the behavior, you
	can turn it off with this code in your Preferences.el / .emacs
	file:

\begin{verbatim} 
(define-key osx-key-mode-map [remap previous-line]
            'previous-line)

(define-key osx-key-mode-map [remap next-line] 
            'next-line)
\end{verbatim}

\item Scrolling and setting the point with the mouse has improved:
	`scroll-margin' defaults to 0 now, so scrolling won't happen until
	you've reached the first or last line in the window - except when
	you use the cursor up/down keys (and osx-key-mode is on). Then,
	scrolling takes place a few lines before you reach the
	boundary. You can configure this via Aquamacs' customization
	variable `visual-scroll-margin'.

\item Command-\{ comments the line or the region (in programming modes)
\item Command-\} uncomments the line or the region (in programming
	modes) 
\item Command-' comments or uncomments (toggling) the line or
	the region (in programming modes)

\item Command-Delete kills (erases) the rest of the current
	line. (Command-Backspace kills the whole line.)
	
\item Filenames with spaces can now be entered without hassle.
	
\item Modifier-Key configuration: Several previously deprecated
	variables have been removed: `mac-reverse-ctrl-meta' and
	'mac-command-key-is-meta'. Use the much simpler, yet
	equally powerful variables `mac-command-modifier', 
	`mac-option-modifier', `mac-control-modifier' and 
	`mac-function-modifier' (experimental) instead.
	Additionally, `mac-pass-option-to-system' is deprecated now:
	instead, set `mac-option-modifier' to nil to reach the same
	effect. The use of the Option menu is recommended. 
	
	To set the Apple key to Meta (not recommended), add this to your
	Preferences.el file:

 	{tt (setq mac-command-modifier 'meta)}

\item Fn-key can act as Meta. If you have an Fn key on your laptop,
	you can use it as Meta key now. Add this to your Preferences.el
	file:

 	{\tt (setq mac-function-modifier 'meta)}

	Note that support of the Fn key is experimental. Rare
	combinations may fail on non-English keyboard layouts,
	and some keys will not work correctly in combination with Fn and Shift.
	 
\item Modifier keys such as Command or Shift show up correctly in the
	shortcuts (menu)

\item New-Buffer menu contains many more modes now, such as modes for
	Fortran, ObjectiveC, C++, PHP, CSS, JavaScript...
	You can add your own in the customization variable
	`aquamacs-known-major-modes'. The menu is updated automatically.

\item Change-Buffer-Mode menu (in File menu) allows to change the mode
	of the current buffer to something hopefully more useful.
	
\item Recently used major modes are now displayed in New-Buffer and
	Change-Buffer-Mode menus .
	
\item M-w (kill-ring-save) will now deactivate the mark after saving
	the region, that is deselect the region, just like on standard
	Emacs. Apple-C will not deselect the region, just like it's done
	on the Mac. 

      \item Run executables from within Emacs that lie in your PATH as
        set in shell init files. Aquamacs will set `exec-path'
        automatically from PATH.  We don't add paths to exec-path any
        more (except one default tetex path for AUCTeX).

\item Questions like "Save to file before closing?" will be asked with
	a proper dialog and not with a menu appearing out of nothing - if
	the last action was carried out with the mouse.           

	Thanks to CarbonEmacsJP project for part of this.

\item Info mode: C-h i  will bring up not only up-to-date info nodes
	for Emacs, but also info on installed packages on the system.
 	(unless you have an INFOPATH environment variable set)

\item Mode-specific styles: With a new menu function, you can now
	apply the style assigned to some other mode, effectively copying
	it over.

\item Mode-specific styles are not applied any more for the major-mode
	of buffers shown in temporary windows inside frames such as the
\item Completions* buffer. This prevents ugly font-switching etc. for
	the whole frame when just a small window is opened inside the frame.

\item New Packages: findr, htmlize, ruby's friends, auctex
 	(latex-mode), nxml-mode, matlab-mode, javascript-mode, 
	
\item Notable Updated external packages: ruby-mode, ruby-*, carbon-font

\item Apple/Command key is now called 'A-' internally because it looks
	like "Apple-" (while it's really the otherwise unused "Alt" key). 
	It used to be "H-" (or: hyper), so change your configuration code
	in case you've defined your own shortcuts.

\item One-Buffer-One-Frame is a minor mode now (called
	one-buffer-one-frame-mode). Use the Options menu (and Save
	Options) to turn it on/off, as before.
	In this minor mode, you have some additional key bindings now:

	C-x B, C-x C-B, C-x S-left, C-x S-right
	they all do the same as their corresponding functions without the
	Shift key, but they will show the buffer that they are switching
	to in the same window, even though one-buffer-one-frame-mode is on.

	The mode has been revised to open fewer extra windows when their
	content is related to a specific frame - for example version
	control checkin comments.
	
\item Manual: The Emacs Manual that comes with Aquamacs (as an Apple
	Help document) has been updated to reflect the basis of Aquamacs,
	namely the upcoming Emacs 22.

\item Command-p-Bugfix: For some users, Emacs locked up at random
	times, showing a message like (wrong command-p ...) in the echo
	area of a frame on every input. This bug in the underlying GNU
	Emacs should be fixed now.
	Patches by Yamamoto Mitsuharu, Richard Stallman, et al.
	
\item Bugfix: Aquamacs now correctly remembers the positions of frames
	with specific files, so they open again on the same screen
	position if the file is re-loaded (in the same session, or after
	"Save Options" in the next session).
\item Bugfix: Bookmarks did not work with `one-buffer-one-frame'
	turned on.  
\item Bugfix: Some Show/Hide options weren't saved by
	"Save Options".
\item Bugfix: links from describe-variable/function help buffers to the 
	source files are correct now for the Aquamacs-specific objects
\item Bugfix: Aquamacs works fine now on systems where, in the default
	shell (e.g. via .bash profile), functions have been defined and
	exported.
	Reported by Terry Jones 
\item Bugfix: RefTeX mode does't give constant audio signals (errors) 
	any more in certain situations
\item Bugfix: Allout mode should work again (saving files)
	Reported by Gerry Brush
	Patch by Ken Mannheimer

\item Donate: Happy about the new version?  Want further improvements?
	Please donate to the Aquamacs project at \url{http://aquamacs.org/donate} -
	Thank you.

\end{itemize}

\subsection{Changes---0.9.6}

\begin{itemize}
\item Aquamacs now provides extended input methods for 
	Korean, Japanese and Chinese (GB/BIG5). You can select them in 
	the Options menu, under "Aquamacs Multilingual Environment" / 
	"Set Language Environment" - everything is integrated with
	Mule.   

	In order to load more fixed-width fonts suitable for use with
	Asian languages, you may want to load the carbon-font package,
	which is included (but not activated) in Aquamacs. Just add the
	following to the file 
	~/Library/Preferences/Aquamacs Emacs/Preferences.el:

	(require 'carbon-font)

	(maintained by   Mahn-Soo Choi.)
	
\item CSS-Mode fixed. 
	(First reported by Kris Khaira and Michael Jakl.)

\item Dual-head setups (two screens) with one screen on the left, 
	the system menu on the right and Aquamacs frames on the left 
	caused problems.	
	(reported by Otto Karl Florian Diesenbacher)
	
\item osx-key-mode-map can now be changed to modify key
	assignments, in particular those of Apple Command keys
	 (implemented in Emacs as Hyper = H- keys.)
	(reported by William Ziemer)

\item "font-lock-keywords ... void" message used to appear when
	attempting to load a new file. This bug in the underlying GNU
	Emacs has been fixed.

      \item Apple-Q and iconized frames: When all frames where
        iconized and one attempted to quit Aquamacs, the application
        seemed to hang (because it was showing a minibuffer prompt in
        an iconized frame.)
		
\item Thanks for the donations!
 
\end{itemize}


\subsection{Interface/Usability Improvements---0.9.5}


\begin{itemize}
\item The application is now called Aquamacs Emacs.



\item The menu of Recent Files will abbreviate long paths now.

	
\item Renamed ``Option key produces only special characters'' to
  ``Option key for Meta (not for extra characters).''

\item C-x C-f  (find-file) will bring up a window first before asking for the
	name of the file to be loaded.

\item Files saved with Aquamacs have a nice file icon now.

\item Text is now nicely shown a little to the right in windows. In
	the left "fringe", you can see small dots indicating the end of
	the buffer, and a small triangle indicating continued lines. If
	you cannot see them, your Frame / default style overwrites
	it. Select "Options / Frame Appearance Styles / Reset all styles"
	to make the fringes appear.

\item The "frame appearance styles" system has been revised. *Messages* now has
	a distinctive colored background. To change this background (or
	the one for *Help* buffers), customize the variable 
	`aquamacs-buffer-specific-frame-styles. Also, you can turn Frame Appearance Styles on and off with a menu
	item.  

\item Setting the frame font gives a warning now (that the font
	is only set for the current frame), or sets it as default if Frame
	Appearance Styles are off.

\item Frame-related functions from File menu are now in Buffers menu.

\item 1on1-* customization variables are gone---most of them were not in working
	order (in Aquamacs) anyways due to the way we were using the
	Oneonone package. This gets rid of quite a bit of doubly
	implemented functionality.

\item You can switch the Option key from its Emacs-specific
	Meta meaning to its Mac typical use (composing keys such as \"{u}
	 (Engl. keyboard) or the backslash (German keyboard) by typing Apple-; -- this
	is equivalent to the Options menu entry ``Option key for Meta.''

\item The customization variable aquamacs-auto-frame-parameters allows you to
	turn off mode-specific styles---your frames stay the same, no
	matter what major mode you switch to. 
 
\item Frame Appearance Styles have a better menu structure now in the Options
	menu. Also, there is a new function ``Delete all mode-specific
	styles''. 

\item Many menu items are now greyed out (cannot be selected) when no frame is visible.

\item Help (Aquamacs Manual) has been revised.


	
\item (Even) speedier startup.



\item Every Aquamacs frame has a toolbar-button now. Use it to turn the
	toolbar in the given frame on and off. 
	

	
\item Improvement when working with multiple frames: when no frame is visible,
	Aquamacs no longer raises some other frame as soon as a file is
	loaded.	



\item Mode-specific styles revised: the full set of parameters coming from 
	color themes is now supported, in addition to the font and toolbar setting.

\item Menu item to delete the mode-specific style for the current major mode.


\end{itemize}



\subsection{New and Updated Major Modes---0.9.5}
\begin{itemize}

\item Better support for editing HTML: HTML-Helper-Mode is included.

\item Emacs Speaks Statistics is included. (Minor glitches with toolbar and  info
    files are known and will be addressed.)

\item Auctex configuration now turns on abbrev-mode and flyspell-mode when
    latex-mode is started (instead of toggling them).
    (reported by Robert Sloan)

\item Page-wise scrolling is more reliable--that is, it is really page-wise and
    you can always return to the same line. (Pager package)
\end{itemize}

\subsection{Bug Fixes---0.9.5}


\begin{itemize}



\item Font ``Bitstream Vera Sans'' works again.  	


\item Exiting view mode (e.g. help) with 'q' deletes the frame instead of
	iconifying it.

\item The (Apple Help) manuals work now on systems where tcsh
	is the default shell.  (Reported by Arthur Ogus)
	

\item Startup with arguments -nw (tty / in terminal) revised.
	


\item You can now redefine Aquamacs keys like this:

	\texttt{(define-key osx-key-mode-map [end] 'end-of-line)}

	(Thanks for reporting: Andreas Hess.)
	
\item If the user changes mac-command-modifier, it may be advisable to call

	 \texttt{(osx-key-reset-mode-map)}

	This will ensure that the keymap is updated and Aquamacs installs
	its Mac-like keyboard commands (provided you want this).

\item New frames will not open behind the Dock any more.


\item An error message (harmless) when clicking on "Finish" in customize is gone.

\item Error message during version check on startup removed (occurred in
	certain cases where users had upgraded their system.)

\item In particular in TeX mode, Aquamacs tended to give audio signals all the
	time when there were internal exceptions (internal errors signaled) with parsing the TeX code. Therefore, we do not ring the bell any more,
	i.e. aquamacs-ring-bell-on-error-flag is nil by default.
	(Reported by Ken Bloom and Andreas Hess)
\item Speedbar works when one-buffer-one-frame (``Open Buffers in Separate
	Frames'') is turned off. 
\item Option-up/down (if Option is meta) now scrolls page-wise.


\item Other bug-fixes.

\end{itemize}
	
\subsection{Bug Fixes---0.9.4}

\begin{itemize}
\item Bug reports contain the subject line that the user entered.

\item No more (delay due to) tramp mode on startup in cases where tramp
    mode had been used before (recentf).

\item Files on externally mounted volumes are not checked for readability
    any more for exclusion from recent files list.
    (Reported by Bill Clementson)

\item No more warnings when fonts are missing (on startup).

\item Drag and Drop while in minibuffer works again.

\item Mouse-2 works again.

\item default-major-mode and initial-major-mode are respected now.
    initial-frame-alist is not modified any more.
    Note that other than in any standard Emacs, mode-specific settings
    always take precedence over default-frame-alist (but not over
    initial-frame-alist).
    (Reported by Rick Zaccone, Joe Davison and Peter Dyballa)

\item No more warnings in customization buffers that the values have
    been changed outside customize (for certain variables).

\item Gnus articles (newsreader) open in same frame now.

\item Gnus has a useful (free) nntp server set so you can start right
    away.

\item Send Email with Sendmail is gone from menu (doesn't usually work
    on  OS X).

\item Fonts corrected: Some glyphs in some fonts turned out wrong when
    in bold or italic. Also, font sizes are correct again (Lucida 14
    is Lucida Grande 14, not 12 or so). The default font size for
    text-mode (and similar ones) is set to Lucida 13 to make up for
    the change.  Apologies for any inconvenience this may cause with
    your default settings. (Reported by William Henney)

\item Better customization for aquamacs-mode-specific-default-themes

\item The customizations file (customizations.el) doesn't grow any more.
    (Reported by Alastair Rankine.)

\item Various corrections in the Aquamacs Manual.

\item Plugin support: All files named site-start.el anywhere in the load-path
    are loaded at startup, before the user's init files are
    loaded. For example, such a file may be placed in
    ~/Library/Application Support/Aquamacs Emacs/myPlugin. This
    feature enables support for easily installable plugins.

\item We anticipate shorter startup times in the next release. Stay tuned.
\end{itemize}


\subsection{Bug Fixes---0.9.3a and 0.9.3b}

\begin{itemize}

\item Fixed ``wrong argument type'' problem (version check),
    which could occur in new installations or those updated from 0.9.1.

\item Monaco 18 font present (suggestion: David M. Cook).

\item Emacs Manual available again.

\item Save mac-pass-option-to-system when saving options.
\end {itemize}


\subsection{Bug Fixes, Features and Changes---0.9.3}

\begin{itemize}

\item Initial-frame-alist is respected. (Reported by Alastair Rankine)


\item When saving a buffer and the Finder is not running, 
	the Finder is not opened any more. (Reported by George W. Gilchrist.)

\item Using dead-keys (like Option-u or Option-n on most
	keyboards) works again. (Reported by Howard Melman and Pierre Albarede
\item When modified files existed, but no frame was visible and
    one tried to quit Aquamacs, the application seemed to hang while
    prompting for keyboard input (Save file? y/n) in an invisible
    window. This has been fixed. 

\item If no frame is visible and you input text, the frame is  made
    visible so you can see what you are doing.

\item Closing windows consistently with the mouse now works properly. 

\item Clicking on links to source files in help buffers properly
    opens a new frame (if ``open buffers in separate frames'' is on).

\item When the buffer shown in a frame changes, sometimes the
    color theme and font were not set correctly; this has been addressed. This could happen when
    ``Show Buffers in Separate Frames'' was off, and one killed a
    buffer. (Reported by Peter Dyballa.)

\item When ``Show Buffers in Separate Frames'' was off, and one killed a
    buffer (kill-buffer, C-x k), the frame was deleted.

\item You have the option to save newly created buffers (Command-N)
    that have not been saved yet when you quit Aquamacs.

\item Less frame-dancing (resizing) on startup.

\item AUC\TeX\ uses the standard commands again. The menu is
  unstructured for this reason. (Reported by Robert Sloan.)

\item No frames could be opened when using a two-screen setup
    with the menu bar on the right screen, and a currently selected
    frame on the left screen. This has been addressed.  (Reported by George W. Gilchrist.)

\item Customizing a style for a special display frame (e.g.  help or
    customization) works now.

\item Recent Files/Clear Menu works again.

\item Paths and other environment variables are now derived
    from the shell  when
    you start Aquamacs. 



\item Your settings for highlight parens,``Blinking Cursor,'' etc.  are now
    persistent. The function Options/Save Options
    (menu-bar-option-save) now correctly saves such settings. This
    should include most customization settings except some that
    Aquamacs relies on for proper clipboard copy/paste functionality.

\item The font menu (Options) no longer includes non-existent fonts. 

\item Option/Show-Hide/Menu Bar is gone because you cannot turn off  the
    menu bar on OS X.

\item ``Send Emacs Bug Report'' now uses the OS X
    default mail program to compose
    a message. This ensures that bug reports actually go through.
    Before, they did not, unless you were running a local SMTP server
    (sendmail/postfix), which is not enabled by default.

\item Fixed loading of files when file names contain certain  Kanji characters,
    due to a bug in AppleScript.

\item The menu shortcut entries were corrected.

\item Characters that require the use of the option key work again. For
    example, Alt-3 produces the pound sign (�) on a US keyboard,
    Alt-L the `at' sign @ and Alt-Shift-7 the backslash $\backslash$ on a
    German one. Inputting the Euro sign works, too. \textit{However, that means that the option key is not used to emulate
    the `meta' modifier;  you will have to use Esc to do that.}
    Alternatively, you can map the option key to meta in your .emacs
    file:

\texttt{    (setq mac-option-modifier 'meta)}

	 \item Improvements in the fontset selection allow you to display the  Euro sign
    with the default font.
     

\item We have moved Aquamacs to \url{http://aquamacs.org}.

 The Aquamacs wiki is now at \url{http://www.emacswiki.org/cgi-bin/wiki/AquamacsEmacs/}.

 The Aquamacs bug reporting address is now \url{mailto:aquamacs-bugs@aquamacs.org}. This can be accessed from directly within the Aquamacs Help menu.

 The general Aquamacs mailing list is still \url{mailto:macosx-emacs@email.esm.psu.edu}; e-mail \url{mailto:macosx-emacs-on@email.esm.psu.edu} to subscribe.
 

\item Slightly faster startup.

\item Runs on OS X 10.3.9, 10.4.0 and 10.4.---tested.

\item Aquamacs automatically checks for updates and notifies the user
    if there is something new. This function communicates with an internet server; it does not
    transmit any information identifying the user. If you would like to
    know more about what is transmitted, use M-x
    aquamacs-check-version-information. If you like to turn this check off, add this to your file
 /Users/yourname/Library/Preferences/Aquamacs Emacs/Preferences.el:

    \texttt{(setq aquamacs-version-check-url nil)}

\item Adding Emacs packages is easier now: Emacs automatically finds
    packages in subdirectories within the /Library/{Preferences| Application
    Support}/{Emacs|Aquamacs Emacs}/ paths.

\item Aquamacs now sets the file creator information of files it
    writes. This helps to open the file from Finder with Aquamacs when
    you double-click it. Feature can be turned off via customization
    option ``aquamacs-set-creator-codes-after-writing-files.''  Also,
    Emacs will appear in the Finder's context menu under ``Open With''
    for a lot of files that it is commonly used to edit.

\item You can start Aquamacs in a terminal (by running
    /Applications/ Emacs.app /Contents/ MacOS/ Emacs) with parameter -nw
    and will show up in the terminal rather than as a  Carbon
    application. Basic file editing and all traditional commands
    work. However, Aquamacs-specific keyboard commands (with the
    Command key) will not work and other functionality may be limited,
    too. \textit{Warning: }This mode of use, which may break in future versions, is not
    supported by the Aquamacs team.
    \end{itemize}
	
	
\item We make sure that the *Completions* buffer (and similar things)
    open as a window inside the frame directly above the Minibuffer,
    and not in a new frame.

\item All newly opened frames open in a somewhat useful position, so
they are not in the way. (If you do not like this, we suggest you set
    your own static frame positions via ``set current style as default''
    and also add this to your file /Users/yourname/Library/Preferences/Aquamacs
    Emacs/Preferences.el:

	\texttt{ (setq smart-frame-positioning-enforce nil)}

Or, if you would like to go with the default position all the time,
    turn the global minor mode off:

   \texttt{(smart-frame-positioning-mode nil)}


\subsubsection{Fonts}
	
\begin{itemize}
	
\item Aquamacs should not complain about missing fonts any more when
    you have upgraded from earlier versions and set scalable fonts as
    default fonts for modes or all frames. They get filtered  automatically.

\item Users with certain setups (cyrillic Lucida Grande) should not get
    a ``default font not found'' error any more.
 

\item ``Recursive Minibuffers'' are enabled. 

\item ``Subscribe to mailing list'' in Help menu.

\item PHP-Mode included (M-x php-mode).

\item Ruby-Mode included (M-x ruby-mode)

\item yes-or-no-p is customizable now. Use the new customization
 	variable aquamacs-quick-yes-or-no-prompt. (Thanks: Pavel Hlavnicka.)

\item Soft word wrap (longlines-mode) is available from the Options
	menu. To make it the default, add this to your preferences
	file:	
	\texttt{(set-default 'longlines-mode t)}

\item Case-insensitive search option has gone into ``search'' submenu (in ``Edit'').

\item If you are in an empty frame (i.e. a frame with an empty buffer)
	and you load (find) a file, Aquamacs will not open an additional frame.
	This is useful also for drag and drops, when a scratch frame is open.

\item The secondary selection is back: use the Command (Apple) key
    together with clicking/dragging the mouse cursor over text in
    order to select text that is not related to the point
    (cursor). This way, you can select text and then scroll somewhere
    else. Extend your selection with shift-command-mouse1.
    To copy/cut the text in the secondary selection to the  clipboard, use
    Shift-Command-C/X, respectively.

\item Some key-bindings are more like the original Emacs ones--in
    particular M-w, which does kill-ring-save again.
    (Idea: Joe Davison)   Also, Home and End keys work as expected.

\item No more annoying system ``ding'' (bell ringing) all the time.  The bell is
    turned off completely, until Emacs developers eliminate the use of
    the bell on user-initiated abort actions (such as ESC ESC ESC when
    in minibuffer, of pressing Cancel in the file selection dialog).
	
\item The toolbar is only displayed in normal frames, but not
    in frames that show help/info buffers. (tool-bar+ and
    aquamacs-tool-bar packages). Turn such toolbars on/off in
    Options/Show/Hide menu.
	
\item The ``About Emacs'' dialogue has been improved.

\item Key combinations with the option key that  involve
    another modifier (that is, ctrl or command) will now work, even
    though simple option combinations are handled by the system to
    produce special characters.

\item The Speedbar is back. Activate in Options/ Show/Hide.

\item The redo function is in the Edit menu now.

\item New buffers (File / New) open in Text Adapt Fill mode now.

\item .save-places and customizations.el do not show up in the recent  files list
    any more.

\item auto-save-files (in / Users / yourname /.emacs.d) are now saved to / Users / yourname/Library/Preferences.

\item ``Save Place in Files in between sessions'' will not generate  files in the
    user's home folder any more. Instead, the file goes into
    / Users / yourname/Library/Preferences/Aquamacs Emacs/ where it belongs.

\item Command-' now cycles between different windows (suggested  by Joseph Kiniry.)



\item Option is mapped to Meta by default, allowing you to enter key
    combinations such as C-M-\ easily. If you'd like to map it to
    alt instead, just add this to your .emacs:

   \texttt{(setq mac-option-modifier 'alt)}
 

\item Color Themes: The color-themes package has been integrated in
    Aquamacs. Use the Option/Color Theme... menu command to choose a
    set of predefined colors for editing source code or writing
    texts. This applies to the current frame only, but you can make it
    the default for all new frames or for all frames in a specific
    mode with the according menu commands.

\item There is a new customization group called ``Aquamacs'' that
    allows you to modify the customizations introduced by Aquamacs.
    This is fairly untested - unexpected results may occur. If so, try
    to locate and fix the bug and send us a patch. If you couldn't
    find the problem, please report via Help/Send Bug Report...
    PLEASE NOTE that a lot of customization variables have changed
    their names---usually, you just need to prepend 1on1  to them.

\item Aquamacs will load your configuration files not just from
    / Users / yourname/.emacs, but also from the location that is appropriate for a Mac
    OS X installation:

/ Library/ Preferences /Aquamacs Emacs / Preferences.el\\
/ Users / yourname/Library/Preferences/Aquamacs Emacs/Preferences.el\\
  /Library/Preferences / Emacs / Preferences.el\\
 / Users / yourname / Library/Preferences/Emacs/Preferences.el\\

    It is recommended to use these instead of / Users / yourname/.emacs on OS X-only
    installations. The first two files should be used for the
    host-wide and user-specific Aquamacs configs, the latter two for
    general Emacs configurations.

\item There is a new configuration option that gives you more fine-grained
    control over how the option modifier key is handled.

\item If mac-pass-option-to system is nil, your Aquamacs will get all key
    combinations. If you press option-3, Aquamacs will see ``M-3'' (or,
    depending on mac-option-modifier ``A-3''). If it is non-nil, you
    will simply get the pound sign (�) on a US keyboard.
 
\item TeXniScope support in LaTeX mode (if installed in /Applications).
 
	\item We have two great manuals available comfortably via Apple Help  now,
    directly from within Aquamacs Emacs (Help menu). There is a
    brand-new Aquamacs manual, and there is Richard
    Stallman's original Emacs manual. They can be searched (e.g. via
    Spotlight on OS X 10.4). We also provide direct access to the online configuration Wiki, which has been filling up with content nicely.  Please contribute; everybody has write access!
	\end{itemize}



\end{document}